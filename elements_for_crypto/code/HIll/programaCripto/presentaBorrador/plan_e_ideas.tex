\documentclass[aspectratio=169]{beamer}
\usepackage[utf8]{inputenc}
\usepackage{amsmath}
\usepackage[T1]{fontenc}
\usepackage[latin1]{inputenc}
\usepackage{pstricks}
\usepackage{graphicx}
\usefonttheme[onlymath]{serif}

\usecolortheme{owl}

\begin{document}


% \begin{frame}
%   \textbf{{\green Plan de labores: C\'atedra calendario. Yuber Hernany Tapias.}}\\
%   \puase

%   \textbf{{\green Actividades de docencia en pregrado:}}
%   \begin{itemize}
%     \item Análisis Numérico (2022-1 y 2022-2)
%     \item Matemáticas y Ciencias Sociales (2022-1 y 2022-2)
%     \item Geometr\'ia Vectorial y álgebra Lineal (2022-1 y 2022-2)
%     \item Fundamentos de Aritmética: Cantidades y Magnitudes (2022-1 y 2022-2)
%   \end{itemize}
% \pause
%   \textbf{{\green Actividades de apoyo intersemestral a docencia en pregrado}}

%   \begin{itemize}
%     \item Acompañamiento al Núcleo Académico de Pensamiento Numérico de Licenciatura de Matemáticas: Revisión de microcurrículos.
%     \item Creación y revisión de microcurrículo o programa de primer nivel de cursos electivos en matemáticas. {\green Modelación y datos 1}.
%     \item Asistencia y apoyo en reuniones ordinarias de Comité de Carrera del pregrado de Lienciatura en Matemáticas.

%   \end{itemize}

% \end{frame}


\begin{frame}
\textbf{Electiva II: {\green Criptograf\'ia.}} Zald\'ivar, Felipe. Introducci\'on a la teor\'ia de n\'umeros. \\ \pause

\begin{itemize}

  \item \textbf{{\green Preliminares}\)} ?`C\'omo resolver problemas aritm\'etico-algebraicos 
  usando c\'alculo simb\'olico? \pause
    \begin{itemize}
        \item Cuadernos de Google Colab. 
        \item Funciones y clases. Módulo sympy.ntheory. \pause
    \end{itemize} 

  \item \textbf{{\green Cifrados de sustituci\'on}} ¿Por qu\'e el algoritmo de la divisi\'on 
  es clave para crear cifrados por sustituci\'on? \pause
      \begin{itemize} 
        \item N\'umeros primos, congruencias y aritm\'etica de residuos. 
        \item Teoremas de Fermat y Euler. Criptosistemas Cesar y Hill. \pause
      \end{itemize} 

  \item \textbf{{\green Primer cifrado de clave p\'ublica}} ¿Favorece al desarrollo del pensamiento computacional el uso
  de atajos en los c\'alculos, usando nuevas representaciones de los n\'umeros enteros? \pause
      \begin{itemize}
        \item Algoritmos para potencias y ra\'ices.
        \item Firmas digitales. Criptosistema RSA. \pause
      \end{itemize}

    \end{itemize}
\end{frame}
\begin{frame}
    \begin{itemize}

    \item \textbf{{\green Ecuaciones m\'odulo un n\'umero entero}} ?`Puede el ordenamiento en tablas de c\'alculos aritm\'eticos fomentar el
    reconocimiento de patrones antes de la formalizaci\'on con teoremas? \pause
        \begin{itemize}
            \item Funciones de Euler y M\"oebius. Ra\'ices. Logaritmos discretos.
            \item Intercambio de claves Diffie-Hellman y Criptosistema ElGamal. \pause
        \end{itemize}

    \item \textbf{{\green Residuos cuadr\'aticos y s\'imbolos de Jacobi}} ?`Practicar criptograf\'ia revela la pertinencia
     de la enseñanza y el aprendizaje de estructuras algebraicas? \pause
        \begin{itemize}
            \item Ley de reciprocidad cuadr\'atica. 
            \item Criptosistema Rabin. \pause
        \end{itemize}

    \item \textbf{{\green Historia reciente de criptograf\'ia}} ?`Es el impacto ambiental la \'unica preocupaci\'on ante la inminente
     creaci\'on de monedas digitales? \pause
        \begin{itemize}
            \item Satoshi Nakamoto: Bitcoin P2P e-cash. 
            \item Activistas digitales Cyberpunk (protección de la privacidad).
            \item Criptomonedas VS Otros servicios en internet. Alto impacto ambiental.\pause
        \end{itemize}

    \end{itemize}

\end{frame}

\end{document}
